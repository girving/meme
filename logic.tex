\documentclass[11pt]{article}
\usepackage{amsfonts,amssymb,amsthm,eucal,amsmath}
\usepackage{graphicx}
\usepackage[T1]{fontenc}
\usepackage{latexsym,url}
\usepackage{array}
\usepackage{subfig}
\usepackage{comment}
\usepackage{color}
\usepackage{tikz}
\usepackage{fancyvrb}
\usepackage{hyperref}
\usepackage[nameinlink,noabbrev]{cleveref}

\hypersetup{colorlinks=true,linkcolor=blue}
\newcommand{\myspace}{\vspace{.1in}\noindent}
\newcommand{\mymyspace}{\vspace{.1in}}
\usepackage[inner=30mm, outer=30mm, textheight=225mm]{geometry}
\creflabelformat{equation}{#2(#1)#3}
\crefname{equation}{}{}
\Crefname{equation}{}{}

\newtheorem{theorem}{Theorem}[section]
\newtheorem{prop}[theorem]{Proposition}
\newtheorem{corollary}[theorem]{Corollary}
\newtheorem{defn}[theorem]{Definition}
\newtheorem{notn}[theorem]{Notation}
\newtheorem{cond}[theorem]{Condition}
\newtheorem{ex}[theorem]{Example}
\newtheorem{rmk}[theorem]{Remark}

\newcommand{\TODO}{{\color{red} TODO}}

% Code and primitives
\newcommand{\cc}[1]{\texttt{#1}}
\newcommand{\Env}{\cc{Env}}
\newcommand{\Known}{\cc{Known}}
%\newcommand{\a}{\&\&}
%\newcommand{\c}[1]{\operatorname{#1}}
\newcommand{\Set}{\operatorname{Set}}
\newcommand{\empty}{\operatorname{empty}}

% Fix weird spacing issue in \cc.  For details, see
% https://tex.stackexchange.com/questions/118455/why-is-there-extra-space-after-in-texttt
%\makeatletter
%\let\ttfamily\relax % avoid a spurious warning
%\DeclareRobustCommand\ttfamily{\not@math@alphabet\ttfamily\mathtt\fontfamily\ttdefault\selectfont\frenchspacing}
%\makeatother

\title{A logic for computation}
\author{Geoffrey Irving\thanks{Email: irving@otherlab.com, Otherlab, San Francisco, CA, United States}}
\date{Version 1, \today}

\begin{document}
\maketitle

We want a logic that mixes computational objects with noncomputational objects, combining the
expressiveness of ZFC with programming.  We'll start by completely ignoring issues of soundness
and universes, then attempt to fix this later.  We will also ignore issues of size (sizeof),
by assuming that all computational types are word sized as in ocaml.  I believe that fixing
these issues will be relatively easy.

More importantly, we start out assuming a purely functional universe.  This assumption is much
harder to eliminate, but we have to start somewhere.

We use ASCII notation where possible to make it easy to translate into code.  In particular,
we use \cc{(:)} in place of $(\in)$ for set membership, \cc{<:} in place of $(\subset)$ for subset,
\cc{->} for function types, and \cc{x => ...} for anonymous functions.

\section{Primitives}

We have two universes \cc{Set} and \cc{Any}, where \cc{Set <: Any}.  \cc{Any} contains both sets
and immutable computational objects (booleans, lists, functions, etc.).  \cc{Set} is the ZFC
universe with individuals, where individuals include at least the computational objects.  We do
not assume the individuals include \emph{only} the computational objects, although that would
be a harmless assumption.

Aside: although \cc{Set} lives in some sense outside the computational universe, we allow
pointers to point directly to sets.  At machine runtime the logic will enforce that such a
pointer is used solely for pointer equality comparison.

The notion of a function is a bit slippery, since we want both the nice properties of ZFC
functions (concrete domain, concrete range, extensionality), and first class support for
partial vs. total and computable vs. noncomputable.  I believe we need two notions; since
the ZFC notion is emergent from sets in ZFC, only the second notion is necessary as a
separate primitive.  We'll curry, so actually what we want is not a notion of function, but
of value.  Since we allow pointers to reference sets, and general values have no properties
that sets don't, this second notion is actually just \cc{Any}.

The primitives in our logic are

\begin{enumerate}
\item $\empty : \Set$: The empty set: \cc{all Any (x => !(x : empty))}
\item \cc{Bool : Set}: The set of booleans
\item \cc{false,true : Bool}: The boolean individuals
\item \cc{(:) : Any}: The set membership function
\item \cc{(==) : Any}: Perfect structural equality over \cc{Any}
\item \cc{(!) : Any}: Boolean not: if \cc{b : Bool}, \cc{!b : Bool} is the other boolean
\item \cc{(\&\&) : Any}: Boolean and
\item \cc{all : Any}: Universal quantification over a set: if \cc{a : Set, f : a -> Bool}, \cc{all f a : Bool}
\item \cc{subsets : Any}: Power set: if \cc{a : Set}, \cc{subsets a : Set}
\item \cc{safe : Any}: Function safety at an argument: if \cc{f : Any, x : Any}, \cc{safe f x : Bool}
\item \cc{filter : Any}: Subset extraction: if \cc{a : Set, f : a -> Bool}, \cc{filter f a : Set}
\item \cc{choice : Any}: Global choice: if \cc{a : Set} is nonempty, \cc{choice a : a}
\end{enumerate}

Important: for now, we explicitly leave out a primitive notion of whether a function is ``computational''
at a value (whether it terminates).  This is because termination is not required for the logic to be
sound, and on a physical machine termination by itself is a useless construct.  Soundness matters, and
efficient execution matters, but little in between.  As with Milawa, we bail at runtime
if the system attempts to run a noncomputational function.

Based on these primitives, we define several derived notions:

\begin{enumerate}
\item Boolean or: \cc{a || b = !(!a \&\& !b)}
\item Function types: \cc{(a -> b) = filter (f => all (x => safe f x \&\& f x : b) a) Any}
\item Existential quantification: \cc{any f a = !(all (x => !(f x)) a)}
\item Function composition: \cc{f . g = (x => f (g x))}
\end{enumerate}

\section{Environments and knowledge}

A proposition is simply a boolean value, together with its definition as code.  Code is an
algebraic datatype
\begin{verbatim}
data Code =
    Var Name
  | Value Any
  | Lambda Name Code
  | Apply Code Code
\end{verbatim}
That's not going to be enough (a bare \cc{Any} is essentially useless, and we've left out
both conditionals and macros), but we'll add further features as necessary.

An environment is a (finite) map from names to values and optionally definitions.  That is
\begin{verbatim}
type Env = Map Name (Any,Maybe Code)
\end{verbatim}
Our knowledge is a set of names that we have proven are true:
\begin{verbatim}
type Known = Set Name
\end{verbatim}

Frequently, we'll have a function in \cc{Env} like \cc{refl = x => x == x}, together with a theorem
of the form \cc{reflAll = all refl Any} which will appear in \cc{Env} and \cc{Known}.  That is, the
theorem fills in all the missing universal quantification in the function.

\section{Definitions and proofs}

At a given point in the code, we have a number of definitions in \Env, some of which are \Known.
We would like to know how to define new things, and how to prove that some of these are true.

\end{document}
